\documentclass{article}
\usepackage[utf8]{inputenc}
\title{DFAT Configurable}
\date{2013-09-01}
\author{
	Heredia Casado, Francisco\\
	\and
	Heredia Casado, Francisco\\
	\and
	Heredia Casado, Francisco\\
	\and
	López García-Ripoll, Diego\\
}
\begin{document}
\maketitle
\newpage
\section{Uso e Interfaz}
Al ejecutar un programa se presenta un menu que a partir del cual se puede
submenú que permite introducir una cadena de caracteres arbitrariamente larga.
Una vez introducida la cadena se informará de si pertenence o no al leguaje y en
caso de que pertenezca se mostrará su traducción.

\section{Modelado del automata}
El autómata consite en un conjunto de estados cada estado guarda
si es final, y el conjunto de transiciones que que salen de el.
Cada transición guarda información del estado al que apunta,
su símbolo y la traducción de dicho símbolo.
Ademas es necesario guardar información del
estado en el que se está y de la traducción que se va formando al ir leyendo la
entrada. Antes de leer ningún caracter de la entrada el estado actual siempre
será el estado inicial.
La forma de operar segun este modelo es la siguiente:
se lee un caracter, se mira si alguna de las transiciones que sale del estado
actual
tiene ese simbolo, si lo tiene se concatena la traducción acumulada con la
traducción del símbolo seído y se guarda el estado destino como estado actual.
Se repite este proceso hasta que no queden simbolos en la cadena de entrada. Por
último se comprueba se el estado actual es final, en cuyo caso se muestra la
traducción.

\section{Implementación}
\subsection{Estructuras dinámicas}
Nos interesa especialmente señalar aquellos elementos cuyo tamaño es
desconocido en tiempo de compilación, que serán los que condicionen nuestra
implementación.
Estados: Se modelan con una estructura, sus direcciones se almacenan en un array en la memoria libre.
Transiciones: Para cada estado se forma una lista enlazada con las transiciones
salientes, guardando la dirección de la primera en el estado en un campo del
estado.
Traducción: se reserva un array de caracteres de tamaño fijo pero en la memoria libre, de
forma que si se llena se puede recolocar con más espacio.
Cadena: No es necesario almacenar la entrada,El autómata lee los caracteres
secuencialmente desde la entrada.

\subsection{Resumen de las estructuras usadas}
Automata: estado en el que se encuentra, si se ha leido correctamente toda la
entrada, la traducción, dirección al array de estados.
Estado: direción de la primera transición de su lista, si es final, el nombre
con el que aparece en el documento.
Transición: su símbolo, su destino, el trozo de traducción asociado, la
dirección de la siguiente transición.

\subsection{Fichero de especificación}
La configuración del autómata consta de 4 partes (ejemplo en el apendice A):
el numero de estados, la lista de estados (los nombres deben estar formados por
carácteres alfanuméricos y tener menos de 20), la lista de los estados finales y
la lista de transiciones TODO ESTO MEJOR COPIARLO DEL README

\end{document}
